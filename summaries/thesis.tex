\documentclass{article}

\begin{document}

\section{Introduction}

Fishery managers have only one ``lever" to pull when it comes to fishery management: the ability to set harvest quotas.  Fishermen work within these quotas by exerting various levels of fishing effort.  Both managers and fisherman can understand the implications of their decisions more easily with the assistance of a production model, which is a system of differential equations designed to predict outcomes.  A visualization enhances such a model by making its inner workings more explicit. We hope to show that our visualization of the MS-PROD model is valuable tool to the modellers and stakeholders alike. %Fishermen work within those quotas by exerting various levels of effort, which is the amount of time spent within a given timeframe to catch a particular species or group of species.  Catchability is defined as that effort as it interacts with the efficiency of their fishing gear and the abundence of the target resource.  Effort and catchability have proven to be key parameters for assessing fisheries \cite{ArreguinSanchez1996}. 

\section{Background}

Both the immediate and long-term effects of human exploitation on fish populations are not easily understood.  There are many other ecological factors to consider besides the harvest, particularly predation and competition.  However, despite recommendations to consider all such factors \cite{united1999ecosystem}, very few models do in the context of fisheries.  Many models tend to leave out some ecological factors on the basis they do not contribute significantly to the decline of fish populations. 

The Lotka-Volterra equations are a set of frequently-used differential equations for describing the interactions between a predator species and a prey species \cite{Lotka1926fds, VOL26a}. Usually, these equations are limited to two species and make assumptions, such as the prey has unlimited food supply and the predator eats the prey exclusively.  

%In addition to the absence of important ecological models, 

\subsection{The Model}

Both the direct and indirect effects of harvests on the various fish populations are not easily understood, especially over time.  The growth or decline of one species can easily affect other species in the ecosystem.  The MS-PROD and Kraken models were created using genetic algorithms fitted to historical data to describe population sizes in terms of harvest, predation, competition, and interaction.

Though these models are useful tools, a more sophisticated interface is necessary to aid in users' understanding of the effects of the various input parameters.  This interface will aid in the comparision of key individual species as well as functional groups.  Part of the interface will include a collection of time series plots representing the modeled biomass of different fish populations over time.  Users will be able to interactively investigate the impact of different levels of effort from fishermen on these populations by controlling effort with sliders, which will result in new time series plots.  Another part of the interface will include network visualizations for the predation, competition, and harvest terms, which would help to illustrate to users which relationships are significant according to the models.  

Many expansions upon the original models are being considered as well, which will be included in the interface and visualization, such as separate levels of effort for the different types of fishing fleets.  The models aim to provide insight into historical fish population sizes and ultimately to estimate the future size of stocks.  Coupled with a well-designed interface, these models will give managers advice on estimated stock levels, which would in turn help them to recommend fishing quotas.

\subsection{Previous Work}

Tweedie et al. developed the Influence Explorer, which is an interface for understanding the relationships between different attributes in a design process \cite{conf/chi/TweedieSDS95}.  Parameters values are randomly selected to represent different possible items.  For each attribute, there is a histogram including each of the items.  The attribute ranges are controlled by sliders.  When the user adjusts the slider of a given attribute, all items that are within that range are highlighted on all of the histograms.  Industrial designers found the ability to interactively explore the effects of different parameter ranges to be valuable.

Ferreira et al. created the BirdVis, a visualization of a statistical model for predicting bird distribution across time and space \cite{Ferreira:6065004}.

Javed et al. studied the merits of different plotting techniques for multiple time series \cite{Javed:2010:GPM:1907651.1907971}.  Their user study revealed that a simple line graph with all time series on one plot or a single graph for each time series is better suited to different tasks than a horizon graph or a braided graph.  They also found that users complete tasks more correctly when there is more display space allocated to the graphs.  A higher number of time series is discouraged because it also leads to a decline in correct task completion.

Knuth illustrated interaction of characters in Victor Hugo's novel \textit{Les Mis\'erables} with an arc diagram \cite{Knuth:1993:SGP:164984}.  Each character is represented with a circular node, where size indicates the number of appearances.  The nodes are arranged linearly, colored and ordered according to clusters of characters that appear together frequently.  Semi-transparent arcs are drawn between the characters which appear in the same chapter, with the thickness of the arc representing the number of such appearances.  %Smaller datasets which contain particular clusters of nodes are well suited to arc diagrams. 

Holten and van Wijk studied the effectiveness of different techniques for indicating directionality of edges in a graph \cite{Holten:2009:USV:1518701.1519054}.  The traditional arrowhead was compared with other representations of direction and was found to perform poorly.  One of their recommendations was that a dark-to-light representation is clearer than light-to-dark for an intensity-based direction cue.

%Wattenberg developed the arc diagram to illustrate repeated sequences in strings or music \cite{Wattenberg:2002:ADV:857191.857733}.  The arc diagram represents the units or significant units of the sequence as nodes that are arranged chronologically in a line.  A translucent arc is drawn between identical units, where the width of the arc corresponds to the length of the repeated sequence, to give a clear indication of pattern and repetition.  Others 

\bibliographystyle{plain}

\bibliography{thesis}

\end{document}


