\documentclass{article}

\begin{document}

\section{Analyzing the tradeoffs among ecological and fishing effects on an example fish community: A multispecies (fisheries) production model}

A multispecies fisheries production model (MS-PROD) \cite{Gamble20092570} incorporates ecological interactions such as predation and competition.  It is distinguished from similiar models by considering competition both within and between groups.  The model itself is a series of differential equations based on the Schaefer production model and takes into account Lotka-Volterra terms for interactions as well as carrying capacities for both the entire system and each group.  The purpose of the model is to evaluate whether or not the modeled ecosystem can support biomass at maximum sustainable yield ($B_{MSY}$).  Simulations done with MS-PROD have implied that it may be impossible for the species in some ecosystems to all be at $B_{MSY}$ at once.  These simulations also emphasized the significance of incoporating predation and competition into production models by showing that these interactions account for a non-trivial amount of the biomass loss.

\section{Summary of June 5, 2013 visit to Woods Hole}

Fishery managers have only one ``lever" to pull when it comes to fishery management: the ability to set harvest quotas.  Fishermen work within those quotas by exerting various levels of effort, which is the amount of time spent within a given timeframe to catch a particular species or group of species.  Catchability is defined as that effort as it interacts with the efficiency of their fishing gear and the abundence of the target resource.  Effort and catchability have proven to be key parameters for assessing fisheries \cite{ArreguinSanchez1996}. 

Both the direct and indirect effects of harvests on the various fish populations are not easily understood, especially over time.  The growth or decline of one species can easily affect other species in the ecosystem.  The MS-PROD and Kraken models were created using genetic algorithms fitted to historical data to describe population sizes in terms of harvest, predation, competition, and interaction.

Though these models are useful tools, a more sophisticated interface is necessary to aid in users' understanding of the effects of the various input parameters.  This interface will aid in the comparision of key individual species as well as functional groups.  Part of the interface will include a collection of time series plots representing the modeled biomass of different fish populations over time.  Users will be able to interactively investigate the impact of different levels of effort from fishermen on these populations by controlling effort with sliders, which will result in new time series plots.  Another part of the interface will include network visualizations for the predation, competition, and harvest terms, which would help to illustrate to users which relationships are significant according to the models.  

Many expansions upon the original models are being considered as well, which will be included in the interface and visualization, such as separate levels of effort for the different types of fishing fleets.  The models aim to provide insight into historical fish population sizes and ultimately to estimate the future size of stocks.  Coupled with a well-designed interface, these models will give managers advice on estimated stock levels, which would in turn help them to recommend fishing quotas.

\bibliographystyle{plain}

\bibliography{summary}

\end{document}


