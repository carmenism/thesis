\section{Questions and Answers} \label{sec:questions}

Our evaluation featured a window which displayed the questions with drop-downs and text boxes for the participants to enter answers.  The questions that were displayed to the participants are below, with correct answers indicated in red italics.

\begin{enumerate}[(A)]
\item Double the harvest effort on \textbf{small pelagics}.

\begin{enumerate}[1.]
\item 
\begin{enumerate}
\item What is the effect on \textbf{herring}? \answerText{Decreased a little.}
\item Why? \answerText{The fishing effort increased, so more herring are caught, resulting in a smaller biomass for herring over time.}
\end{enumerate}
\end{enumerate}

\item Halve the harvest effort on \textbf{flatfish}.

\begin{enumerate}[1.] \addtocounter{enumii}{1}
\item 
\begin{enumerate}
\item What is the effect on \textbf{winter flounder}? \answerText{Increased a lot.}
\item Why? \answerText{Winter flounder is a flatfish, so fishing less for flatfish results in an increased biomass for winter flounder.}
\end{enumerate}

\item 
\begin{enumerate}
\item What is the effect on \textbf{yellowtail flounder}? \answerText{Stayed about the same.}
\item Why? \answerText{One would expect the biomass of yellowtail flounder to increase due to halving the flatfish fishing effort.  However, both windowpane and winter flounder, both flatfish, compete with yellowtail flounder.  Their increased biomass seems to have prevented the yellowtail flounder biomass from increasing much.}
\end{enumerate}
\end{enumerate}

\item Double the harvest effort on \textbf{elasmobranchs}.

\begin{enumerate}[1.] \addtocounter{enumii}{3}
\item 
\begin{enumerate}
\item What is the effect on \textbf{skates}? \answerText{Decreased a lot.}
\item Why? \answerText{Skates are elasmobranchs, so fishing more for elasmobranchs resulted in a decrease in biomass.}
\end{enumerate}

\item 
\begin{enumerate}
\item What is the effect on \textbf{cod}? \answerText{Increased a lot.}
\item Why? \answerText{Spiny dogfish, which are elasmobranchs, predate on cod.  Doubling the harvest on elasmobranchs caused the biomass of spiny dogfish to decrease.  Since there were less spiny dogfish to predate on the cod, the cod biomass increased.}
\end{enumerate}

\item 
\begin{enumerate}
\item What is the effect on \textbf{haddock}? \answerText{Decreased a lot.}
\item Why? \answerText{Spiny dogfish, which are elasmobranchs, predate on cod.  The decrease in elasmobranchs caused an increase in cod.  Cod compete with haddock, so the increase in cod led to a decrease in haddock.}
\end{enumerate}

\item 
\begin{enumerate}
\item What is the effect on \textbf{windowpane}? \answerText{Decreased a lot.}
\item Why? \answerText{Spiny dogfish predate on cod, so the increased harvest on elasmobranchs led to the cod biomass increasing.  Cod compete with windowpane, so the windowpane biomass suffers due to the increased cod biomass. }
\end{enumerate}
\end{enumerate}

\end{enumerate}

\subsection{Difficulty Levels}

The ``Why?'' questions vary in terms of difficulty, based on the order of the effects:

\begin{itemize}
\item \textbf{First order:} the biomass of the fish changed directly as a result of increased or decreased harvest effort.  In other words, the fish was a member of the functional group whose harvest effort was changed and either (a) the biomass decreased due to increased harvest effort or (b) the biomass increased due to decreased harvest effort.

E.g.: increase harvest on A $\rightarrow$ biomass of A decreases.
\begin{itemize}
\item Question 1
\item Question 2
\item Question 4
\end{itemize}
\item \textbf{Second order:} the biomass of the fish changed as a result of the harvest effort of another fish changing.  The other fish predates on or competes with the first fish and its biomass changed because it was harvested more or less.

E.g.: increase harvest on A $\rightarrow$ biomass of A decreases $\rightarrow$ biomass of B increases because A eats B.
\begin{itemize}
\item Question 3
\item Question 5
\end{itemize}
\item \textbf{Third order:} the explanation for the change in biomass involves two other fish species.  The three species are involved with each other due to competition and/or predation and only one of the species changed in biomass due to being harvested more or less.

E.g.: increase harvest on A $\rightarrow$ biomass of A decreases $\rightarrow$ biomass of B increases because A eats B $\rightarrow$ biomass of C decreases because B eats C.
\begin{itemize}
\item Question 6
\item Question 7
\end{itemize}
\end{itemize}