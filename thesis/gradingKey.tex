\chapter{Evaluation Grading Scheme}

\section{Note to Graders}

\begin{itemize}
\item The ``What is the effect on \_\_\_\_\_?'' questions are graded automatically.
\item The ``Why?'' questions are graded on a scale of 0 to 3 (partial points allowed).  Examples and explanations of the answers are listed below.
\item Assign grades without considering the condition the participant was assigned. 
\end{itemize}

\section{Explanation}

\rowcolors{2}{gray!10}{gray!3}
\begin{tabular}{| c | l |} \hline
\rowcolor{gray!35} \textbf{Score} & \textbf{Meaning} \\ \hline
0 & Completely wrong \\ 
1 & More wrong than right \\ 
2 & More right than wrong \\ 
3 & Completely correct \\
\hline
\end{tabular}


\clearpage

\section{Grading Key}

{\setlength{\itemsep}{2em}
\begin{enumerate}


\item 
\begin{enumerate}[i.]
\item What is the effect on \textbf{herring}? 

{\small
\rowcolors{2}{cyan!15}{cyan!5}
\begin{tabular}{| l | l |} \hline
\rowcolor{cyan!35} \textbf{Answers} & \textbf{Score} \\ \hline
Decreased a lot & 2 \\ 
Decreased a little & 3 \\ 
Stayed about the same & 1 \\ 
Increased a little & 0 \\
Increased a lot & 0 \\
\hline
\end{tabular}
}

\item Why?

{\small
\rowcolors{2}{cyan!15}{cyan!5}
\begin{tabular}{| l | p{5.25cm} | p{5.7cm} |} \hline
\rowcolor{cyan!35} \textbf{Score} & \textbf{Example} & \textbf{Description} \\ \hline
3 & Herring are small pelagics, so they are being caught more when small pelagic fishing effort increases, therefore their biomass decreases. & Mentions that herring are being \textbf{caught more} (since herring are small pelagics). \\ 
2 & More small pelagics are being caught. & Mentions small pelagic fishing effort increased without indicating that this means more herring were being caught. \\ 
1 & We are fishing for herring. & Generic statement like ``Harvesting increased'' or something which implies we were not fishing for herring before. \\ 
0 & Since we are harvesting more small pelagics, there are less for the herring to eat. & Something false, confusing, irrelevant, etc. \\
\hline
\end{tabular}
}

\end{enumerate}

\clearpage

\item 
\begin{enumerate}[i.]
\item What is the effect on \textbf{winter flounder}?

{\small
\rowcolors{2}{red!15}{red!5}
\begin{tabular}{| l | l |} \hline
\rowcolor{red!35} \textbf{Answers} & \textbf{Score} \\ \hline
Decreased a lot & 0 \\ 
Decreased a little & 0 \\ 
Stayed about the same & 0 \\ 
Increased a little & 1 \\
Increased a lot & 3 \\
\hline
\end{tabular}
}

\item Why?

{\small
\rowcolors{2}{red!15}{red!5}
\begin{tabular}{| l | p{5.25cm} | p{5.7cm} |} \hline
\rowcolor{red!35} \textbf{Score} & \textbf{Example} & \textbf{Description} \\ \hline
3 & Winter flounder are flatfish, so they are being caught less due to the decreased harvest effort of flatfish, therefore their biomass increases. & Mentions that winter flounder are being \textbf{caught less} (since winter flounder are flatfish). \\ 
2 & Less flatfish are being caught. & Mentions flatfish fishing effort decreased without indicating that this means more winter flounder were being caught.
 \\ 
1 & Harvest decreased. & Generic statement that is true but does not make a conclusion. \\ 
0 & We are not fishing for winter flounder.
 & Something false, confusing, irrelevant, etc. \\
\hline
\end{tabular}
}

\end{enumerate}

\clearpage

\item 
\begin{enumerate}[i.]
\item What is the effect on \textbf{yellowtail flounder}?

{\small
\rowcolors{2}{red!15}{red!5}
\begin{tabular}{| l | l |} \hline
\rowcolor{red!35} \textbf{Answers} & \textbf{Score} \\ \hline
Decreased a lot & 0 \\ 
Decreased a little & 0 \\ 
Stayed about the same & 3 \\ 
Increased a little & 2 \\
Increased a lot & 0 \\
\hline
\end{tabular}
}

\item Why? 

{\small
\rowcolors{2}{red!15}{red!5}
\begin{tabular}{| l | p{5.25cm} | p{5.7cm} |} \hline
\rowcolor{red!35} \textbf{Score} & \textbf{Example} & \textbf{Description} \\ \hline
3 & Winter flounder and windowpane, the other flatfish, both compete with yellowtail flounder.  They both grew because of the decreased effort on flatfish.  Their increase in biomass led to increased competition on yellowtail flounder, which prevented growth of the yellowtail flounder. & Mentions that windowpane and winter flounder \textbf{compete} with yellowtail flounder.  Mentions their biomass increased, which made the effects of competition more pronounced.  \\ 
2 & It stayed about the same because there was a lesser effort in catching fish, and because both compete with yellowtail flounder.
 & Blames competition without clearly saying who is competing with the yellowtail flounder or mentions that some fish compete with the yellowtail flounder without mentioning that those fish had an increase in biomass. \\ 
1 & It appears that the yellow flounder is not affected as much by a change in fishing rates. & Generic statement that is true but does not make a conclusion. \\ 
0 & Yellowtail and haddock populations both remained stable, so the balance was unchanged. & Something false, confusing, irrelevant, etc. \\
\hline
\end{tabular}
}

\end{enumerate}

\clearpage

\item 
\begin{enumerate}[i.]
\item What is the effect on \textbf{skates}?

{\small
\rowcolors{2}{violet!15}{violet!5}
\begin{tabular}{| l | l |} \hline
\rowcolor{violet!35} \textbf{Answers} & \textbf{Score} \\ \hline
Decreased a lot & 3 \\ 
Decreased a little & 1 \\ 
Stayed about the same & 0 \\ 
Increased a little & 0 \\
Increased a lot & 0 \\
\hline
\end{tabular}
}

\item Why?

{\small
\rowcolors{2}{violet!15}{violet!5}
\begin{tabular}{| l | p{5.25cm} | p{5.7cm} |} \hline
\rowcolor{violet!35} \textbf{Score} & \textbf{Example} & \textbf{Description} \\ \hline
3 & Skates are elasmobranchs.  Since the harvest effort increased on elasmobranchs, skates are being harvested more and their biomass decreased. & Mentions that skates are being \textbf{caught more} (since skates are small pelagics).  \\ 
2 & We are harvesting more elasmobranchs. & Mentions elasmobranch fishing effort increased without indicating that this means more skates were being caught. \\ 
1 & We doubled the harvest effort. & Generic statement like ``Harvest increased'' or ``We are fishing for skates'' (implying that we were not fishing for them before). \\ 
0 & Skates compete with spiny dogfish. & Something false, confusing, irrelevant, etc. \\
\hline
\end{tabular}
}

\end{enumerate}

\clearpage

\item 
\begin{enumerate}[i.]
\item What is the effect on \textbf{cod}?

{\small
\rowcolors{2}{violet!15}{violet!5}
\begin{tabular}{| l | l |} \hline
\rowcolor{violet!35} \textbf{Answers} & \textbf{Score} \\ \hline
Decreased a lot & 0 \\ 
Decreased a little & 0 \\ 
Stayed about the same & 0 \\ 
Increased a little & 1 \\
Increased a lot & 3 \\
\hline
\end{tabular}
}

\item Why?

{\small
\rowcolors{2}{violet!15}{violet!5}
\begin{tabular}{| l | p{5.25cm} | p{5.7cm} |} \hline
\rowcolor{violet!35} \textbf{Score} & \textbf{Example} & \textbf{Description} \\ \hline
3 & Spiny dogfish, which are elasmobranchs, predate on cod. Doubling the harvest on elasmobranchs caused the biomass of spiny dogfish to decrease. Since there were less spiny dogfish to predate on the cod, the cod biomass increased. & Mentions that spiny dogfish, which are being caught more, \textbf{predate} on cod, which leads to an increase in cod. \\ 
2 & Spiny dogfish decreased, so cod increased. & Mentions spiny dogfish are significant without mentioning predation. \\ 
1 & Spiny dogfish predate on cod. & Generic, truthful statement that doesn't have an argument or conclusion in it. \\ 
0 & Because they don't hunt each other. & Something false, confusing, irrelevant, etc. \\
\hline
\end{tabular}
}

\end{enumerate}

\clearpage

\item 
\begin{enumerate}[i.]
\item What is the effect on \textbf{haddock}?

{\small
\rowcolors{2}{violet!15}{violet!5}
\begin{tabular}{| l | l |} \hline
\rowcolor{violet!35} \textbf{Answers} & \textbf{Score} \\ \hline
Decreased a lot & 3 \\ 
Decreased a little & 2 \\ 
Stayed about the same & 0 \\ 
Increased a little & 0 \\
Increased a lot & 0 \\
\hline
\end{tabular}
}

\item Why?

{\small
\rowcolors{2}{violet!15}{violet!5}
\begin{tabular}{| l | p{5.25cm} | p{5.7cm} |} \hline
\rowcolor{violet!35} \textbf{Score} & \textbf{Example} & \textbf{Description} \\ \hline
3 & Spiny dogfish, which are elasmobranchs, predate on cod. The decrease in elasmobranchs caused an increase in cod. Cod compete with haddock, so the increase in cod led to a decrease in haddock. & Mentions that spiny dogfish, which are being caught more, \textbf{predate} on cod, which leads to an increase in cod.  Cod \textbf{compete} with haddock, so haddock decease. \\ 
2 & There is more competition with cod due to there being less due to less spiny dogfish, which predate on cod. & Slightly less precise language or is missing some details, but mentions all of the key species involved. \\ 
1 & There are more cod so there are less haddock. & Missing a lot of the details, mentions at least one of the relevant species. \\ 
0 & Haddock populations decreased a little because spiny dogfish and skates compete with each other. & Something false, confusing, irrelevant, etc. \\
\hline
\end{tabular}
}

\end{enumerate}

\clearpage

\item 
\begin{enumerate}[i.]
\item What is the effect on \textbf{windowpane}?

{\small
\rowcolors{2}{violet!15}{violet!5}
\begin{tabular}{| l | l |} \hline
\rowcolor{violet!35} \textbf{Answers} & \textbf{Score} \\ \hline
Decreased a lot & 3 \\ 
Decreased a little & 1 \\ 
Stayed about the same & 0 \\ 
Increased a little & 0 \\
Increased a lot & 0 \\
\hline
\end{tabular}
}

\item Why?

{\small
\rowcolors{2}{violet!15}{violet!5}
\begin{tabular}{| l | p{5.25cm} | p{5.7cm} |} \hline
\rowcolor{violet!35} \textbf{Score} & \textbf{Example} & \textbf{Description} \\ \hline
3 & Spiny dogfish, which are elasmobranchs, predate on cod. The decrease in elasmobranchs caused an increase in cod. Cod compete with haddock, so the increase in cod led to a decrease in haddock. & Mentions that spiny dogfish, which are being caught more, \textbf{predate} on cod, which leads to an increase in cod.  Cod \textbf{compete} with windowpane, so windowpane decease. \textit{OR} Mentions that winter flounder and yellowtail flounder both are \textbf{preyed} on by elasmobranchs.  Since elasmobranchs decrease, yellowtail flounder and winter flounder increase.  Both \textbf{compete} with windowpane, so windowpane suffers. \\ 
2 & There is more competition with cod due to there being less due to less spiny dogfish, which predate on cod. & Slightly less precise language or is missing some details, but mentions all of the key species involved. \\ 
1 & There are more cod so there are less windowpane. & Missing a lot of the details, mentions at least one of the relevant species but doesn't make a coherent argument. \\ 
0 & There were no fish for them to eat. & Something false, confusing, irrelevant, etc. Mentions that the skates compete with windowpane. \\
\hline
\end{tabular}
}

\end{enumerate}
\end{enumerate}


}