\chapter{Conclusion}

A long-term goal of this thesis has been to help the fishery management community make informed decisions with the use the MS-PROD model through our interface.  Declaring whether we achieved this long-term goal requires public unveiling of the MS-PROD model by its original authors and time to determine if it actually benefits managers, fishermen, and other significant stakeholders, therefore no conclusion can be drawn yet for this goal.  On the other hand, the short-term goal of this thesis has been develop an interactive visualization which would enhance understanding of a complicated, multi-species production model, MS-PROD.  More specifically, we investigated the effectiveness of different methods for depicting causal relationships.  To measure this objectively, we conducted an evaluation of the different depictions of the predation and competition relationships between the ten key species of the MS-PROD model.

\section{Future Work}

There are many possible directions this research could take in the future.  Most immediately, the visualization could be adapted to newer versions of the MS-PROD model which incorporate more ecological factors, such as climate, and more realistic simulations of harvest, which include many types of fishing gear and catchabilities for each fish species for each fishing gear.  Future versions of MS-PROD may also include spatial information, which would also require further expansion of our visualization.  Researchers at NOAA's National Marine Fisheries Services also expressed interest in adapting the visualization to other models.  We implemented our visualization to be as agnostic to the model as possible, but more work might be necessary to allow for easy addition of new features and to allow other models to be used in place of MS-PROD.  However, a more generic version of the visualization would be a powerful tool which many scientists, modelers, and ecologists might be interested in using.

Much more work remains concerning the visualization of causal relationships, as there are many other possible depictions which we did not explore.  With dynamic arcs, we were attempting to encode for causal relationships that might explain differences between forecasts.  Different equations to determine the widths of these dynamic arcs could have been investigated.  Furthermore, the arcs themselves could have been drawn differently to more precisely describe the differences between forecasts---e.g., colors or drawing styles could have been varied to show how the arc grew or shrank instead of using plus signs or minus signs.  More evaluations could be conducted to compare new designs with the designs for arcs presented in this thesis.

Finally, Dr.\ Michael Fogerty and Robert Gamble---who developed the MS-PROD model with Jason Link---are interested in adapting the visualization to be more accessible by the general public.  The most simple way to do this would be to allow users to download an executable of the visualization from the web which could be run on a local machine.  However, a more effective---though time-consuming---approach would be to develop a version of the visualization which can easily be run in a browser, because casual users might be less inclined to download an executable.  This second approach would require reimplementation in a browser-friendly language such as JavaScript.
