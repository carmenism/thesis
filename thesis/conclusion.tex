\chapter{Conclusion}

A long-term goal of this thesis has been to help the fishery management community make informed decisions with the use the MS-PROD model through our interface.  Declaring whether we achieved this long-term goal requires public unveiling of the MS-PROD model by its original authors and time to determine if it actually benefits managers, fishermen, and other significant stakeholders, therefore no conclusion can be drawn yet for this goal.  On the other hand, the short-term goal of this thesis has been develop an interactive visualization which would enhance understanding of a complicated, multi-species production model, MS-PROD.  More specifically, we investigated the effectiveness of different methods for depicting causal relationships.  Our informal interviews with expert users led us to believe we had succeeded in creating a visualization to their satisfaction.  However, to measure this objectively, we conducted an evaluation of the different depictions of the predation and competition relationships between the ten key species of the MS-PROD model.

Our evaluation of the depictions of the inter-species relationships showed that having arcs is superior to no arcs for answering higher-order questions about changes in biomass.  This was not surprising, since no arcs is uninformative and the higher-order questions involved complex, indirect effects from changes in fishing effort.  A visualization of the arcs was essential to answer these types of questions.  More investigation is necessary to find significant differences between the different types of arcs---i.e., static, dynamic, and animated dynamic. 

Additionally, our visualization contributes to the community of modeling in general.  Our interface is a novel combination of time series, a network diagram, interactions, and the model itself.  These combined elements allow users to not only compare differences between two forecasts, but also to understand what factors account for these differences, leading to a better understanding of the model in general.  The ability to correctly interpret a model is crucial for users of all levels of expertise, since models can easily become complicated as more factors are incorporated in order to make the model more accurate.  We believe our visualization approach might be a powerful tool for other ecological models or even models in other fields such as economics, medicine, engineering, and so on.

\section{Future Work}

Most immediately, we are curious to see if a larger study of the different depictions of the causal relationships would yield more significant results.  In particular, we would hope to see a major difference between static and dynamic arcs.  Besides recruiting more participants and perhaps restructuring the questions to be asked, we could also explore alternative depictions of the causal relationships.  With dynamic arcs, we were attempting to encode for causal relationships that might explain differences between forecasts.  Different equations to determine the widths of these dynamic arcs could have been investigated.  Furthermore, the arcs themselves could have been drawn differently to more precisely describe the differences between forecasts---e.g., colors or drawing styles could have been varied to show how the arc grew or shrank instead of using plus signs or minus signs.

Concerning the model itself, there are many possible directions this research could take in the future.  The visualization could be adapted to newer versions of the MS-PROD model which incorporate more ecological factors, such as climate, and more realistic simulations of harvest, which include many types of fishing gear and catchabilities for each fish species for each fishing gear.  Future versions of MS-PROD may also include spatial information, which would also require further expansion of our visualization.  Researchers at NOAA's National Marine Fisheries Services also expressed interest in adapting the visualization to other models.  The visualization was implemented to be as agnostic to the model as possible, but more work might be necessary to allow for easy addition of new features and to allow other models to be used in place of MS-PROD.  However, a more generic version of the visualization would be a powerful tool which many scientists, modelers, and ecologists might be interested in using.

More work would be required in order to increase the scalability of the visualization.  A model might represent an ecosystem of twenty species, which might require a whole new approach since arc diagrams are better suited to smaller datasets.  Perhaps a traditional node-link diagram could be used where the node for each fish species is a compact time series chart.  For much larger datasets (i.e. fifty or more), the user would need a mechanism to view subsets of the data at once to avoid being overwhelmed by the entire dataset, such as zooming, pop-up windows, a fisheye navigation, or an ``expand and collapse'' feature.

Finally, Dr.\ Michael Fogerty and Robert Gamble---who developed the MS-PROD model with Jason Link---are interested in adapting the visualization to be more accessible by the general public.  The most simple way to do this would be to allow users to download an executable of the visualization from the web which could be run on a local machine.  However, a more effective---though time-consuming---approach would be to develop a version of the visualization which can easily be run in a browser, because casual users might be less inclined to download an executable.  This second approach would require reimplementation in a browser-friendly language such as JavaScript.  For either approach, both Dr.\ Fogerty and Gamble are reluctant to allow for actual species names to be shown in case the results are over-interpreted.  They recommend that more abstract names such as ``Small Pelagic 1''  or ``Elasmobranch 2'' be used.
