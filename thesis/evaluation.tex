\chapter{Evaluation}

The intention behind this work has been to develop an interactive visualization that effectively portrays the MS-PROD model and its implications.  More specifically, we are interested in how different visualization alternatives enhance a user's understanding of the complex relationships between the fish species and their effects.  In other words, is there a benefit to using dynamic, animated arcs---the most complicated representation of the relationships---over another method or even displaying no arcs at all?  To investigate this question, we designed and conducted a user study to the performance of different arc depiction alternatives.

\section{Method}

The study was conducted at a screened-off table in a student union building at the University of New Hampshire.  A paid undergraduate research assistant conducted the study.  For each participant, the research assistant explained the MS-PROD model and our visualization, and showed a training example before leaving the participant to the experiment.

\section{Participants}

There were 80 participants who took part in the study, all of which were recruited by a poster affixed to the backside of the privacy screen.  Participants were voluntary and were compensated with a pack of pens or a notebook.  They were required to read and sign an IRB consent form before participating in the study.

\section{Apparatus}

We conducted the experiment using a standard Dell laptop with an extra Dell monitor.  The window with the model visualization was maximized on the extra screen, while the window with the experiment questions was maximized on the laptop screen.  Participants used the mouse to interact with the model visualization and recorded their answers using the laptop keyboard.

\section{Task}

Participants were first instructed as follows:


