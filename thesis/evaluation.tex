\chapter{Evaluation}

The intention behind this work has been to develop an interactive visualization that effectively portrays the MS-PROD model and its implications.  We conducted two types of evaluations to determine which design alternatives were best for conveying this information: a formal evaluation of the arcs with novice users and an informal evaluation with expert users.  These two evaluations and the resulting findings are described in details below. 

\section{Formal Evaluation of Dynamic Arcs}

We were interested in how different visualization alternatives enhance a user's understanding of the complex relationships between the fish species and the effects of those relationships.  In other words, is there a benefit to using dynamic, animated arcs---the most complicated representation of the relationships---over another method or even displaying no arcs at all?  To investigate this question, we designed and conducted a user study to the performance of different arc depiction alternatives.

Our hypothesis was that the condition which features no arcs would be the least effective, as it requires the user to guess why indirect or unexpected changes in biomass occured.  We also hypothesized that the dynamic arcs would be more effective than the static arcs, because dynamic arcs filter to show the relevant information, while static arcs show all information at once which could be overwhelming.  Finally, we hypothesized that the animated, dynamic arcs would be at least slightly more effective than the non-animted, dynamic arcs, since they have more visual cues for directionality.

\subsection{Method}

The study was conducted at a screened-off table in a student union building at the University of New Hampshire.  A paid undergraduate research assistant conducted the study.

The research assistant explained the MS-PROD model and our visualization, and showed a training example before leaving the participant to the experiment.  Feedback was received only during the traiing phase.  A single experiment lasted approximately ten minutes.

Responses from the study were graded by two paid undergraduate research assistants.

\subsection{Participants}

There were 80 participants who took part in the study, all of which were recruited by a poster affixed to the backside of the privacy screen.  Participants were voluntary and were compensated with a pack of pens or a notebook.  They were required to read and sign an IRB consent form before participating in the study.

\subsection{Apparatus}

We conducted the experiment using a standard Dell laptop with an extra Dell monitor.  The window with the model visualization was maximized on the extra screen, while the window with the experiment questions was maximized on the laptop screen.  Participants used the mouse to interact with the model visualization and recorded their answers using the laptop keyboard.

\subsection{Experimental Conditions}

Each participant conducted the experiment task for only one of the four conditions:

\begin{enumerate}[(A)]
\item No arcs
\item Static arcs
\item Dynamic arcs without animation
\item Dynamic arcs with animation
\end{enumerate}

Explanations and training phases were tailored according to the experimental condition---e.g., arcs were explained only for conditions B, C, and D; dynamic arcs were explained only for conditions C and D.

\subsection{Task}

Initially, all fishing effort sliders were set to the value of one.  Participants were  instructed to increase or decrease the fishing effort of a specific functional group, e.g.:
\begin{quote}
\textit{Using the sliders, double the harvest effort on elasmobranchs.}
\end{quote}

Next, the participants were asked to answer one or more questions of the form, ``\textit{What was the effect on (fish species)?}''  The questions were designed so that sometimes the fish species was a member of the function group for which the effort was just adjusted, while other times the fish species was not a member of that functional group, e.g.:
\begin{quote}
\textit{What is the effect on haddock?}
\end{quote}
Users answered this question with one of five options from a drop-down menu:
\begin{itemize}
\item \textit{Increased a lot}
\item \textit{Increased a little}
\item \textit{Stayed about the same}
\item \textit{Decreased a little}
\item \textit{Decreased a lot}
\end{itemize}

Finally, the user was asked, ``\textit{Why? [Try to explain in no more than three sentences.]}''  A large text box was provided for the participant to type a response.  If this question was the last question in its set, then the sliders are all reset to one and a new instruction is given for the next set of questions. 

In total, there were three instructions and eight questions.  All participants were given the same instructions and asked the same questions in the same order, regardless of condition.

\subsection{Results}

ABC

\section{Informal Evaluation}
