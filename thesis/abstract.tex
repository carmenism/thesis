\begin{abstractpage} 

Fishery management is the science of setting rules for governing fishing so that it is done in a sustainable manner.  An ecosystem-based fisheries management (EBFM) approach has been advocated to recognize ecosystems as the complex systems that they are.  If EBFM is to be put into effect, then fishery managers require ecological models which take many factors into account. MS-PROD is one such model; it is a multi-species production model which forecasts the biomass of ten species of fish in the Gulf of Maine over 30 years.  In the model, the biomass of each fish species depends on effects from harvesting and interactions with the other fish species.  An interactive visualization to the model was designed and implemented to allows the user to investigate the impact of changes in fishing effort in real time.  By combining time series with a network representation, this visualization shows the predicted biomasses of the fish, the changes in biomass that can result from changes in fishing effort, the causal relationships that help to explain the effects of changes in fishing effort, and the uncertainty of the model.  An evaluation was conducted to compare four different methods for depicting the causal relationships in the model---predation and competition---which found that representing those relationships with arc diagrams enhanced understanding of the model.  This visualization is a novel combination of features which may be a powerful tool for fishery managers to use to make informed decisions.

\end{abstractpage}