\chapter{Work to be Completed}
%and then creating a finalized user interface for the MS-PROD model by selecting from the different visualization alternatives according to the evaluation results.
The remaining work consists mostly of completing a user evaluation and then finalizing the interactive interface based on the results of the evaluation. Some minor implementation changes may also be needed to support the evaluation.  The purpose of the evaluation is to assess the design alternatives with the intended goal of creating a user interface to the MS-PROD model.%When finalized, our user interface for the MS-PROD model should achieve its goal of being an effective tool for models and other stakeholders to use to evaluate fishery quota decisions.

\section{User Evaluation}

There will be two phases to the user evaluation.  First, experts will be instructed on how to use the MS-PROD visualization and questioned about their preferences in the form of a semi-structured interview.  Second, novice users will be evaluated on their understanding of the model.

\subsection{Feature Comparison by Experts}

For the expert evaluation, we will visit the authors of the MS-PROD model, scientists, fishery managers, and other significant stakeholders.  First, we will instruct them on how to use the various visualization alternatives.  Then, we will ask them about their preferences and opinions regarding the two views (four panel grid and small multiples), the three methods for displaying change (instantaneous change only, line, and blend), and the four types of uncertainty (multi-line, box plots, error bars, and error bands).  The different features will be either ranked or rated.  Some questions which may be asked are:

\begin{itemize}
  \item Which visualization method did you like the best?
  \item Is this feature useful?
\end{itemize}

\subsection{Goal of Model Understanding}

The second evaluation will be aimed at a more objective evaluation of how well the arc and change representations enable people to understand the underlying model.  This will require undergraduate or graduate students as participants.  Participants will be instructed in how to use the interactive interface to the model.  Next, participants will be asked questions to guage their understanding of the model.  For example, the participant may asked questions like, ``What happens to X if we increase catch on Y?''  More importantly, this question will be followed up with, ``Why?''  The purpose of these questions is to determine whether or not average users can gain an understanding of the complex relationships that result in indirect and chained effects.
